The Higgs boson was added to the Standard Model (SM) ten years ago, and since then physicists have been remeasuring and using the Higgs as an integral part of research.  
Using the Higgs boson to probe for new physics phenomena is part of the high energy experimental frontier. 
At CERN the Compact Muon Solenoid (CMS) gathers data from proton proton collisions to push this frontier further. 
In this dissertation a search is conducted for exotic decays of the Standard Model Higgs Boson, $H$, decaying to a pair of pseudoscalars, $a$, which then decay to pairs of muons and tau leptons. 
This search supports many beyond Standard Model (BSM) theories which could solve the $\mu$ coupling problem in Super Symmetry and fit into Axion-like Models (Peccei-Quinn) and Grand Unified Theories (GUTs). 
Due to the model independent nature of the search, the SM is also tested within the kinematic range.  
Pseudoscalar masses between 20 and 60 GeV are investigated using the full Run II dataset collected at CMS, corresponding to a luminosity of 137 $\text{fb}^{-1}$. The existence of the pseudoscalar Higgs is primarily motivated by Two Higgs Doublet Models with the extension of a Singlet (2HDM+S). Upper limits on the branching fraction are set.
