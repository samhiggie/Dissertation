This Ph.D culminates almost 10 years at CERN with CMS. Along the way I've met some truly astounding researchers, scientists, mentors, and friends.  

Firstly, I would like to acknowledge Daniel Marlow, my advisor, confidant, and mentor. Dan's philosophical stance on the Ph.D has been freedom with genuine support. His generosity, guidance, and flexibility has been nothing less than profound. Even after working with him closely for six years, I still learn from his wisdom. 

Prof. Christopher Tully has also been supportive, honest, and caring during my time at Princeton. I want to express my gratitude for his role in reviewing the thesis as the reader and the support he has given the Princeton CMS group with living overseas. 

Special thanks to my committee Prof. Lisanti and Prof Leifer for agreeing to review and conduct the final public oration on short notice and during the busy summer time.  

Additionally, I want to thank some more senior researchers that supported my work in the lab and in this analysis. Andre Frankenthal and Alexis Kalogeropoulos were helpful in guidance and providing tools to conduct the Higgs-like search. Christopher Palmer and Jingyu Luo were influential on luminosity operations and beyond. In the lab, Fengwangdong Zhang, Alan Honma, and Alessandro La Rosa made outer tracker research fun and captivating. 

Going back, the start of my high energy physics journey was in Indiana. The curiosity I had at the discovery of the Higgs in 2012 guided my interest. At Purdue University, I had the pleasure of jumping into research right after the Higgs discovery by working with Daniela Bortoletto, Kirk Arndt, Gale Lockwood, and Nick Hinton. It was the best introduction to CMS research anyone could have! They welcomed me into the CMS family and pushed me all the way to Princeton. 

I want to thank my friends at Princeton who welcomed a wandering Hoosier. Seth Olsen, was my first friend at Princeton and the best party planner I know. His honesty, intensity, and love of the Rolling Stones knows no bounds. He is an inspiration in adapting, changing, and overcoming. 

Luckily, I got to experience CERN with some Princeton folks too. I made fond memories around St. Genis with Gage DeZoort, Nick Haubrich, Bennett Greenberg, Stephanie Kwan, and Gillian Kopp. I have faith that this group will push the frontier of the field forward! 

After the summer, I found myself as the only Princeton graduate student staying and living overseas during the covid pandemic. Dylan Teague,  Grace Cummings, Andrés Delannoy, Janina Nicolini, Kevin Sedlaczek, Anders Mikkelsen, Arild Velure, Bingxuan Liu, Tal van Daalen, Syed Haider Abidi,  Matthias Weber, Vivan Nguyen, and Jeff Shahinan comprise my CERN family. They were my world when the rest of the world shut down. I'm truly grateful for their company. 

Dylan Teague deserves special recognition. Not only was he my main editor for this dissertation, but he also taught me more software, analysis, and physics techniques than anyone else. He is a devote friend and confidant. His appetite to learn is only matched by his vigor to climb. He has my highest regard. 
\newline
\newline
\newline
\newline
\newline

\textit{... To Mom, Dad, and my dear Sophie.}  

