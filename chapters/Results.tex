
%results

After all the event selection, including the signal extraction cuts listed in section \ref{sec:selection} the statistical hypothesis test can be made. 
The final number of events listed in each of the categories for the full Run II dataset are shown in the table below \ref{tab:event_yield}


\begin{table}[h!tbp]
\centering
\topcaption{Expected event yields of signal an background across all years
\label{tab:event_yield}
}
\begin{tabular*}{0.8\textwidth}{c|c|c}
\hline
Signal & \multicolumn{2}{c}{Background} \\
\hline $m_a=40\text{GeV}$ & Data Driven (FF) & Irreducible (ZZ)\\\hline
6.54   & 23.61 & 6.33 \\\hline
\end{tabular*}
\end{table}
 



Each unbinned parametric likelihood fit is included for each mass point into a confidence level scan to set an upper limit at 95\% on the branching ratio. Initial values of the signal distributions are selected to make sure that the signal strength modifier in the limit is of order unity. The range of masses reflected are shown between 20 GeV and 60 GeV to ensure compatibility with Higgs aa combination limits for more exotic Higgs models --- like those at lower $a$ mass. 
In order to estimate the upper limit at 95\% CL on the branching fraction a simple Poisson model can be used. For a statistically limited search we can choose to estimate the background yeild as no events. The upper limit on branching fraction calculated earlier: 
\[B =  \frac{N}{\sigma \cdot A\cdot \mathcal{L}} = 0.00043\] 
An upper limit of 95\% is set on the branching fraction by adjusting the signal strength (event yield) until a p-value of 5\% is reached on the joint likelihood function for the fit model. 
The event yield is normalized with a set branching fraction, which was assumed to be the Higgs $\sigma_{SM} \times 0.01\%$.
Multiplying the CL by 0.01\% thus yields the limit on $\frac{\sigma_h}{\sigma_{SM}} B(h\rightarrow aa\rightarrow2\mu2\tau)$.
Preliminary limits are set using the asymptotic limit method ~\cite{Cowan_2011} for each mass point.


\begin{figure}[ht!b]
  \centering
  \includegraphics[width=0.47\textwidth]{Figures/CLs/plotLimit_aa_2016_mmet_2016.png}
  \includegraphics[width=0.47\textwidth]{Figures/CLs/plotLimit_aa_2016_mmtt_2016.png}\\
  \includegraphics[width=0.47\textwidth]{Figures/CLs/plotLimit_aa_2016_mmmt_2016.png}
  \includegraphics[width=0.47\textwidth]{Figures/CLs/plotLimit_aa_2016_mmem_2016.png}\\
    \caption{\label{fig:CLs2016} Asymptotic CL Limits on the branching fraction times ratio of the SM cross sections}
\end{figure}

\begin{figure}[ht!b]
  \centering
  \includegraphics[width=0.47\textwidth]{Figures/CLs/plotLimit_aa_2017_mmet_2017.png}
  \includegraphics[width=0.47\textwidth]{Figures/CLs/plotLimit_aa_2017_mmtt_2017.png}\\
  \includegraphics[width=0.47\textwidth]{Figures/CLs/plotLimit_aa_2017_mmmt_2017.png}
  \includegraphics[width=0.47\textwidth]{Figures/CLs/plotLimit_aa_2017_mmem_2017.png}\\
    \caption{\label{fig:CLs2017} Asymptotic CL Limits on the branching fraction times ratio of the SM cross sections for 2017}
\end{figure}

\begin{figure}[ht!b]
  \centering
  \includegraphics[width=0.47\textwidth]{Figures/CLs/plotLimit_aa_2018_mmet_2018.png}
  \includegraphics[width=0.47\textwidth]{Figures/CLs/plotLimit_aa_2018_mmtt_2018.png}\\
  \includegraphics[width=0.47\textwidth]{Figures/CLs/plotLimit_aa_2018_mmmt_2018.png}
  \includegraphics[width=0.47\textwidth]{Figures/CLs/plotLimit_aa_2018_mmem_2018.png}\\
    \caption{\label{fig:CLs2018} Asymptotic CL Limits on the branching fraction times ratio of the SM cross sections for 2018}
\end{figure}

All of the years and channels are then combined to form the combined result and the model 2HDM+S interpretations for different scenarios. Scenario type III where coupling to $\tau$ leptons is favored is expected to be the most stringent scenario for this final state. 

\begin{figure}[ht!b]
  \includegraphics[width=0.65\textwidth]{Figures/combination/plotLimit_aa_2022_all_2022_zoom.png}
    \caption{\label{fig:CLsRunII} Asymptotic CL Limits on the branching fraction times ratio of the SM cross sections for the full Run II dataset $\text{137}\text{fb}^{-1}$}
\end{figure}

\begin{figure}[ht!b]
  \centering
  \includegraphics[width=0.47\textwidth]{Figures/combination/2HDM_Model_2DLimits_I.png}
  \includegraphics[width=0.47\textwidth]{Figures/combination/2HDM_Model_2DLimits_II.png}\\
  \includegraphics[width=0.47\textwidth]{Figures/combination/2HDM_Model_2DLimits_III.png}
  \includegraphics[width=0.47\textwidth]{Figures/combination/2HDM_Model_2DLimits_IV.png}\\
    \caption{\label{fig:2HDM} 2HDM model specific scenarios upper 95\% CL limits on the branch fraction of Higgs to a a times the ratio of the SM cross sections for the full Run II dataset $\text{137}\text{fb}^{-1}$ }
\end{figure}

